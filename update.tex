\documentclass[11pt,a4paper]{article}

% \documentclass{article}
% \usepackage{arxiv}

\usepackage{authblk}

\usepackage[utf8]{inputenc} % allow utf-8 input
\usepackage[T1]{fontenc}    % use 8-bit T1 fonts
\usepackage{hyperref}       % hyperlinks
\usepackage{url}            % simple URL typesetting
\usepackage{booktabs}       % professional-quality tables
\usepackage{amsfonts}       % blackboard math symbols
\usepackage{nicefrac}       % compact symbols for 1/2, etc.
\usepackage{microtype}      % microtypography
%\usepackage{cleveref}       % smart cross-referencing
%\usepackage{lipsum}         % Can be removed after putting your text content
\usepackage{graphicx}
\usepackage{natbib}
\usepackage{doi}

\usepackage{comment}

% Uncomment to override  the `A preprint' in the header
% \renewcommand{\headeright}{}
% \renewcommand{\undertitle}{}
%\renewcommand{\shorttitle}{Integrated Hessian}

%\documentclass[final,times,5p,7pt,twocolumn,sort&compress,fleqn]{elsarticle}
%\usepackage[numbers]{natbib}

\pretolerance=2000

% to correct space before and after delimiters
\usepackage{mleftright}
\mleftright 

%page geometry
% \usepackage[colorlinks,linkcolor=IVAIteal,citecolor=IVAIpurple,urlcolor=IVAIpurple]{hyperref}
 % \usepackage[lmargin=2cm,rmargin=2cm,tmargin=1.25cm,bmargin=1.25cm,includefoot,includehead]{geometry}

%typography
% \usepackage[T1]{fontenc}
% \usepackage[utf8]{inputenc}
%% Serif Times fonts
%\renewcommand{\rmdefault}{ptm}

\usepackage[xspace]{ellipsis}
% \usepackage{pifont}% http://ctan.org/pkg/pifont
% \newcommand{\cmark}{\ding{51}}%
% \newcommand{\xmark}{\ding{55}}%

% \usepackage{charter}

\usepackage{inconsolata} % for tighter \tt (monospace fonts)

\usepackage{xfrac}

% \usepackage{titling}
% \pretitle{\begin{center}\Large\fontfamily{phv}\fontseries{bf}\selectfont}
% \posttitle{\end{center}}
% \preauthor{\begin{center}\large}
% \postauthor{\end{center}}
% \predate{\begin{center}\footnotesize}
% \postdate{\end{center}}
% \setlength{\droptitle}{-1cm}


%\usepackage{charter}

%% Sans-serif Arial-like fonts (Helvetica)
%\renewcommand{\sfdefault}{phv} 

%text structures
\usepackage[inline,shortlabels]{enumitem}
\usepackage{tabu}
\usepackage[dvipsnames]{xcolor}
\usepackage{tcolorbox}
\usepackage{colortbl}
\usepackage{graphicx}
\usepackage{eurosym}
\usepackage{xspace}
\usepackage{booktabs}

\usepackage{relsize}

\usepackage{subcaption}
\captionsetup{font=footnotesize}

\usepackage{float}
%\usepackage[parfill]{parskip}
%\usepackage{enumitem}
\usepackage{multirow}
\usepackage{tabulary}

% \usepackage{mathrsfs}
% \usepackage{mathtools}

\usepackage{amsmath,amsthm}
\interdisplaylinepenalty=2500

\usepackage{amssymb}
%\usepackage{bbm}
\usepackage{bm}
%\usepackage{upgreek}

\usepackage{ifthen} % provides \ifthenelse test  
\usepackage{xifthen}

\usepackage[active]{srcltx}

\usepackage{stmaryrd} % provides double brackets

\usepackage{xspace}

\DeclareMathOperator*{\diag}{diag}

\renewcommand{\arraystretch}{1.3} 

\newcommand\undermat[2]{%
  \makebox[0pt][l]{$\smash{\underbrace{\phantom{%
    \begin{bmatrix}#2\end{bmatrix}}}_{\text{$#1$}}}$}#2}

\makeatletter
\DeclareRobustCommand\onedot{\futurelet\@let@token\@onedot}
\def\@onedot{\ifx\@let@token.\else.\null\fi\xspace}
\def\eg{{e.\,g}\onedot} \def\Eg{{E.g}\onedot}
\def\ie{{i.\,e}\onedot} \def\Ie{{I.e}\onedot}
\def\cf{{cf}\onedot} \def\Cf{{C.f}\onedot}
\def\etc{{etc}\onedot}
\def\vs{{vs}\onedot}
\def\wrt{w.\,r.\,t\onedot}
\def\dof{d.\,o.\,f\onedot}
\def\etal{{et al}\onedot}
\def\almoste{a.\,e\onedot}
\makeatother

% to stretch bmatrices, pmatrices, and the like
% \makeatletter
% \renewcommand*\env@matrix[1][\arraystretch]{%
%   \edef\arraystretch{#1}%
%   \hskip -\arraycolsep
%   \let\@ifnextchar\new@ifnextchar
%   \array{*\c@MaxMatrixCols c}}
% \makeatother

% \usepackage{array}
% \renewcommand{\arraystretch}{1.5}

%\usepackage{cmdtrack}
%\usepackage{refcheck}

% \newcommand*{\expect}{\mathsf{E}}
% \newcommand*{\prob}{\mathsf{P}}
% \newcommand{\IG}{\Xib_{\mathrm{IG}}}
% \newcommand{\Psib}{\bm{\Uppsi}}
% \newcommand{\Vm}{V^-}
% \newcommand{\Vpm}{V^{\pm}}
% \newcommand{\Vp}{V^+}
% \newcommand{\Win}{\M{W}_{\textrm{in}}}
% \newcommand{\Wout}{\M{W}_{\textrm{out}}}
% \newcommand{\Xib}{\bm{\Upxi}}
% \newcommand{\Zplus}{\mathbb{Z}_{\geqslant{0}}}
% \newcommand{\agm}[5][f]{\tilde{\mu}^{(#1)}_{#2,#3;#4,#5}}
% \newcommand{\gm}[5][f]{\mu^{(#1)}_{#2,#3;#4,#5}}
% \newcommand{\lambdab}{\bm{\uplambda}}
% \newcommand{\lambpm}{\lambda_{\pm}}
% \newcommand{\pcr}{\tens{C}}
% \newcommand{\phibt}{\tilde{\bm{\upphi}}}
% \newcommand{\phib}{\bm{\upphi}}
% \newcommand{\phit}{\tilde{\phi}}
% \newcommand{\pro}[2][P]{\M{#1}_{#2}^\perp}
% \newcommand{\psib}{\bm{\uppsi}}
% \newcommand{\tde}{\tilde\de}
% \newcommand{\truede}{\underline{\de}}
% \newcommand{\truetde}{\underline{\tde}}
%\newcommand{\IG}{{\bm{\upgamma}}}
%\newcommand{\J}{\mathbf{J}}
%\newcommand{\betabold}{\boldsymbol{\upbeta}}
%\newcommand{\xib}{\bm{\xi}}
\newcommand*\diff{\mathop{}\!\mathrm{d}}
\newcommand{\0}{\M{0}}
\newcommand{\IGs}{u}
\newcommand{\IG}{\bm{\nabla}^{\mathrm{int}}}
\newcommand{\IH}{\bm{\mathsf{D}\nabla}^{\mathrm{int}}}
\newcommand{\M}[1]{\mathbf{#1}}
\newcommand{\Mt}[1]{\tilde{\M{#1}}}
\newcommand{\Mr}[1]{{\M{#1}}^{\mathrm{red}}}
\newcommand{\Mrt}[1]{{\tilde{\M{#1}}}^{\mathrm{red}}}
\newcommand{\N}{\mathbb{N}}
\newcommand{\Z}{\mathbb{Z}}
\newcommand{\Nzero}{\Z^+}
\newcommand{\R}{\mathbb{R}}
\newcommand{\T}{\top}
\newcommand{\comp}[2]{{\left\llbracket #1\right\rrbracket}_{#2}}
\newcommand{\dderiv}[2]{\dfrac{\partial #1}{\partial #2}}
\newcommand{\deq}{\mathrel{\stackrel{\scriptscriptstyle\Delta}{=}}}
\newcommand{\deriv}[2]{\frac{\partial #1}{\partial #2}}
\newcommand{\der}[1]{\bm{\mathsf{D}}{#1}}
\newcommand{\de}{\boldsymbol{\theta}}
\newcommand{\emphbf}[1]{\emph{\textbf{#1}\xspace}}
\newcommand{\entry}[3]{{\left\llbracket#1\right\rrbracket}_{#2 #3}}
\newcommand{\e}{\mathrm{e}}
\newcommand{\grad}[1]{\bm{\nabla}{#1}}
\newcommand{\hess}[1]{\bm{\mathsf{D}\nabla}{#1}}
\newcommand{\sderiv}[2]{{\partial #1}/{\partial #2}}
\newcommand{\ud}{\,\mathrm d}
\newcommand{\ve}[1]{\mathbf{#1}}
\newcommand{\eve}[2]{\mathbf{e}^{(#1)}_{#2}}

\newcommand{\derivtwo}[3]{%
  \ifthenelse{\equal{#2}{#3}}
  {\frac{\partial^2#1}{\partial #2^2}}
  {\frac{\partial^2#1}{\partial #2 \partial #3}}
}

\newcommand{\dderivtwo}[3]{%
  \ifthenelse{\equal{#2}{#3}}
  {\dfrac{\partial^2#1}{\partial #2^2}}
  {\dfrac{\partial^2#1}{\partial #2 \partial #3}}
}

% \newcommand{\dderivtwo}[3]{%
%   \ifthenelse{\isempty{#3}}
%   {\dfrac{\partial^2#1}{\partial #2^2}}
%   {\dfrac{\partial^2#1}{\partial #2 \partial #3}}
% }

\DeclareMathOperator*{\argmax}{arg\,max}
\DeclareMathOperator*{\argmin}{arg\,min}
\DeclareMathOperator{\sgn}{sgn}

\usepackage[active]{srcltx}

\usepackage{colonequals}


% generating plots

\usepackage{pgfplots}
\pgfplotsset{compat=newest}

\setlist{nosep} 

\newtheorem{theorem}{Theorem}%[section]
\newtheorem*{theorem*}{Theorem}

\newenvironment{changemargin}[2]{%
  \begin{list}{}{%
      \setlength{\topsep}{0pt}%
      \setlength{\leftmargin}{#1}%
      \setlength{\rightmargin}{#2}%
      \setlength{\listparindent}{\parindent}%
      \setlength{\itemindent}{\parindent}%
      \setlength{\parsep}{\parskip}%
    }%
  \item[]}
  {\end{list}}

\newenvironment{keywords}{%
  \begingroup
  \def\and{\unskip\space\textperiodcentered\space\ignorespaces}
  \begin{changemargin}{\leftmargin}{\leftmargin}
    \small\noindent\emph{Keywords}:}
  {\end{changemargin}
  \endgroup
}

\begin{document}

\title{A Note on the Incremental SVD}

\date{}

\author{}

% \affil{%
%   Insight Via Artificial Intelligence Pty Ltd \\ Suite 811, 147 Pirie Street \\ Adelaide, SA 5000}

\maketitle

% \begin{abstract}
%   The mathematical formulation of the integrated Hessian method from machine learning involves a matrix whose entries contain integral expressions. Here we present a formula for the integrated Hessian matrix that uses single integrals over a line segment. This replaces the original expression for the same matrix that uses double integrals over a square domain.
% \end{abstract}

% \begin{keywords}
%   integrated gradient \and integrated Hessian \and single integral \and double integral
% \end{keywords}

%\title{A Note on the Integrated Hessian}

%\maketitle

\section{Updating}

\subsection{The Iterative Process}

Let
\begin{displaymath}
  \M{X}_0 = \M{U}_0 \M{S}_0 \M{V}_0^\T.
\end{displaymath}
Then
\begin{displaymath}
  \M{X}_1
  = [\M{U}_0, \ve{p}_1] \M{K}_1
  \begin{bmatrix}
    \M{V}_0 & \0
    \\
    \0^T & 1
  \end{bmatrix}^\T
  =
  [\M{U}_0, \ve{p}_1] \M{C}_1 \M{S}_1 \M{D}_1^\T
  \begin{bmatrix}
    \M{V}_0 & \0
    \\
    \0^T & 1
  \end{bmatrix}^\T.
 \end{displaymath}
With
\begin{displaymath}
  \M{U}_1 =  [\M{U}_0, \ve{p}_1] \M{C}_1
  \quad
  \text{and}
  \quad
  \M{V}_1 =
  \begin{bmatrix}
    \M{V}_0 & \0
    \\
    \0^T & 1
  \end{bmatrix}
  \M{D}_1,
\end{displaymath}
we have
\begin{displaymath}
  \M{X}_1 =   \M{U}_1 \M{S}_1 \M{V}_1^\T.
\end{displaymath}
Generally, for any $k \in \N$, 
\begin{displaymath}
  \M{X}_k
  =
  [\M{U}_{k-1}, \ve{p}_k] \M{K}_k
  \begin{bmatrix}
    \M{V}_{k-1} & \0
    \\
    \0^T & 1
  \end{bmatrix}^\T
  =
  [\M{U}_{k-1}, \ve{p}_k] \M{C}_k \M{S}_k \M{D}_k^\T
   \begin{bmatrix}
    \M{V}_{k-1} & \0
    \\
    \0^T & 1
  \end{bmatrix}^\T,
\end{displaymath}
and with
\begin{displaymath}
  \M{U}_k =  [\M{U}_{k-1}, \ve{p}_k] \M{C}_k
  \quad
  \text{and}
  \quad
  \M{V}_k
  =
  \begin{bmatrix}
    \M{V}_{k-1} & \0
    \\
    \0^T & 1
  \end{bmatrix}
  \M{D}_k,
\end{displaymath}
we have
\begin{displaymath}
  \M{X}_k =  \M{U}_k \M{S}_k \M{V}_k^\T.
\end{displaymath}

\subsection{Updating the $\M{U}$ matrices}

Note that, for each $k \in \N$, 
\begin{align*}
  \M{U}_{k+1}
  &
    = [\M{U}_k, \ve{p}_{k+1}] \M{C}_{k+1}
    =  [[\M{U}_{k-1}, \ve{p}_k] \M{C}_k, \ve{p}_{k+1}] \M{C}_{k+1}
  \\
  &
    =
    \left[
    [\M{U}_{k-1}, \ve{p}_k], \ve{p}_{k+1}
    \right]
    \begin{bmatrix}
      \M{C}_k & \0
      \\
      \0^\T & 1
    \end{bmatrix}
              \M{C}_{k+1}.
\end{align*}
This equality inspires us to introduce, inductively, the following matrices:
\begin{displaymath}
  \Mt{U}_1 = [\M{U}_0, \ve{p}_1]
  \quad
  \text{and}
  \quad
  \Mt{C}_1 = \M{C}_1,
\end{displaymath}
and, for each positive integer $k$ greater than 1, 
\begin{displaymath}
  \Mt{U}_k
  =
  [ \Mt{U}_{k-1},  \ve{p}_k]
  \
  (
  = 
  [\M{U}_0, \ve{p}_1, \dots, \ve{p}_k]
  )
  \quad
  \text{and}
  \quad
  \Mt{C}_k
  =
  \begin{bmatrix}
    \Mt{C}_{k-1} & \0
    \\
    \0^\T & 1
  \end{bmatrix}
  \M{C}_k.
\end{displaymath}
We claim that
\begin{equation}
  \label{eq:1}
  \M{U}_k = \Mt{U}_k \Mt{C}_k
\end{equation}
for each $k \in \N$.  Indeed, the claim holds vacuously for $k = 1$. Assume that the claim holds for $k$. Then, since
\begin{displaymath}
  \Mt{U}_{k+1} = [\Mt{U}_k, \ve{p}_{k+1}]
  \quad
  \text{and}
  \quad
  \Mt{C}_{k+1}
  =
  \begin{bmatrix}
    \Mt{C}_k & \0
    \\
    \0^\T & 1
  \end{bmatrix}
  \M{C}_{k+1},
\end{displaymath}
we have
\begin{displaymath}
  \Mt{U}_{k+1} \Mt{C}_{k+1}
  =
  [\Mt{U}_k, \ve{p}_{k+1}]
  \begin{bmatrix}
    \Mt{C}_k & \0
    \\
    \0^\T & 1
  \end{bmatrix}
  \M{C}_{k+1}
  \\
  =
  [\Mt{U}_k  \Mt{C}_k, \ve{p}_{k+1}] \M{C}_{k+1}.
\end{displaymath}
By the inductive hypothesis, expressed as \eqref{eq:1}, 
\begin{displaymath}
  [\Mt{U}_k  \Mt{C}_k, \ve{p}_{k+1}] \M{C}_{k+1}
  =
  [\M{U}_k, \ve{p}_{k+1}] \M{C}_{k+1}.
\end{displaymath}
But
\begin{displaymath}
  [\M{U}_k, \ve{p}_{k+1}] \M{C}_{k+1}
  =
  \M{U}_{k+1}
\end{displaymath}
by definition.  Thus
\begin{displaymath}
  \M{U}_{k+1} = \Mt{U}_{k+1} \Mt{C}_{k+1},
\end{displaymath}
and so the claim holds for $k+1$.

\subsection{Updating the $\M{V}$ matrices}

Note that, for each $k \in \N$, 
\begin{align*}
  \M{V}_{k+1}
  &
    =
    \begin{bmatrix}
      \M{V}_k & \0
      \\
      \0^T & 1
    \end{bmatrix}
             \M{D}_{k+1}
             =
             \begin{bmatrix}
               \begin{bmatrix}
                 \M{V}_{k-1} & \0
                 \\
                 \0^T & 1
               \end{bmatrix}
               \M{D}_k & \0
               \\
               \0^T & 1
             \end{bmatrix}
                      \M{D}_{k+1}
  \\
  & =
    \begin{bmatrix}
      \begin{bmatrix}
        \M{V}_{k-1} & \0
        \\
        \0^T & 1
      \end{bmatrix}
      & \0
      \\
      \0^T & 1
    \end{bmatrix}
             \begin{bmatrix}
               \M{D}_k & \0
               \\
               \0^\T & 1
             \end{bmatrix}
                       \M{D}_{k+1}.
\end{align*}
This equality suggests that we introduce, inductively, the following matrices:
\begin{displaymath}
  \Mt{V}_1
  =
  \begin{bmatrix}
    \M{V}_0 & \0
    \\
    \0^\T & 1
  \end{bmatrix}
  \quad
  \text{and}
  \quad
  \Mt{D}_1 = \M{D}_1,
\end{displaymath}
and, for each positive integer $k$ greater than 1, 
\begin{displaymath}
  \Mt{V}_k
  =
  \begin{bmatrix}
    \Mt{V}_{k-1} & \0
    \\
    \0^\T & 1
  \end{bmatrix}
            \
            \left(=
            \begin{bmatrix}
              \M{V}_0 & \0 & \0 & \0 & \cdots & \0
              \\
              \0^\T & 1 & 0 &  0 & \cdots & 0
              \\
              \0^\T & 0 & 1 & 0 & \cdots & 0
              \\
              \vdots & \vdots & \vdots & \ddots & & \vdots 
              \\
              \vdots & \vdots & \vdots & & \ddots & \vdots 
              \\
              \0^\T & \undermat{\text{$k$ columns}}{0 & 0 & 0 & \cdots & 1}
            \end{bmatrix}
                                           \right)
\end{displaymath}
and
\begin{displaymath}
  \Mt{D}_k
  =
  \begin{bmatrix}
    \Mt{D}_{k-1} & \0
    \\
    \0^\T & 1
  \end{bmatrix}
  \M{D}_k.
\end{displaymath}
We claim that
\begin{equation}
  \label{eq:2}
  \M{V}_k = \Mt{V}_k \Mt{D}_k
\end{equation}
for each $k \in \N$.  Indeed, the claim holds vacuously for $k = 1$. Assume that the claim holds for $k$. Then, since
\begin{displaymath}
   \Mt{V}_{k+1}
  =
  \begin{bmatrix}
    \Mt{V}_k & \0
    \\
    \0^\T & 1
  \end{bmatrix}
  \quad
  \text{and}
  \quad
  \Mt{D}_{k+1}
  =
  \begin{bmatrix}
    \Mt{D}_k & \0
    \\
    \0^\T & 1
  \end{bmatrix}
  \M{D}_{k+1},
\end{displaymath}
we have
\begin{displaymath}
  \Mt{V}_{k+1} \Mt{D}_{k+1}
  =
   \begin{bmatrix}
    \Mt{V}_k & \0
    \\
    \0^\T & 1
  \end{bmatrix}
  \begin{bmatrix}
    \Mt{D}_k & \0
    \\
    \0^\T & 1
  \end{bmatrix}
  \M{D}_{k+1}
  \\
  =
  \begin{bmatrix}
     \Mt{V}_k \Mt{D}_k & \0
    \\
    \0^\T & 1
  \end{bmatrix}
  \M{D}_{k+1}.
\end{displaymath}
By the inductive hypothesis, expressed as \eqref{eq:2}, 
\begin{displaymath}
  \begin{bmatrix}
     \Mt{V}_k \Mt{D}_k & \0
    \\
    \0^\T & 1
  \end{bmatrix}
  \M{D}_{k+1}
  =
  \begin{bmatrix}
     \M{V}_k & \0
    \\
    \0^\T & 1
  \end{bmatrix}
  \M{D}_{k+1}.
\end{displaymath}
But
\begin{displaymath}
  \begin{bmatrix}
     \M{V}_k & \0
    \\
    \0^\T & 1
  \end{bmatrix}
  \M{D}_{k+1}
  =
  \M{V}_{k+1}
\end{displaymath}
by definition.  Thus
\begin{displaymath}
  \M{V}_{k+1} = \Mt{V}_{k+1} \Mt{D}_{k+1},
\end{displaymath}
and so the claim holds for $k+1$.

\subsection{Updating when the rank does not increase}

\subsubsection{Description}

For each $n \in \N$, let $r_n$ denote the rank of $\M{K}_n$. Suppose that $k \in \N$ is such that $r_{k+1} = r_k$. Then the unreduced SVD of $\M{K}_{k+1}$ takes the form
\begin{displaymath}
  \M{K}_{k+1} = \M{C}_{k+1} \M{S}_{k+1} \M{D}^\T_{k+1},
\end{displaymath}
where
\begin{align*}
  \M{C}_{k+1}
  & =
  \begin{bmatrix}
    \M{C}^{1:r_k, 1:r_k}_{k+1} & \0
    \\
    \0^\T & 1
  \end{bmatrix},
  \\[2ex]
  \M{S}_{k+1}
  & =
    \diag(s_{1,k+1}, \dots, s_{r_k,k+1}, 0),
  \\[2ex]
  \M{D}_{k+1}
  & =
    \begin{bmatrix}
      \M{D}^{1:r_k, 1:r_k}_{k+1} & \M{D}^{1:r_k, r_k+1}_{k+1}
      \\
      \M{D}^{r_k+1, 1:r_k}_{k+1} & \M{D}^{r_k+1, r_k+1}_{k+1}
    \end{bmatrix}.
\end{align*}
The matrix $\M{K}_{k+1}$ admits also an SVD in reduced form, namely
\begin{displaymath}
  \M{K}_{k+1} = \Mr{C}_{k+1} \Mr{S}_{k+1} (\Mr{D}_{k+1})^\T,
\end{displaymath}
where
\begin{align*}
  \Mr{C}_{k+1}
  & =
  \begin{bmatrix}
    \M{C}^{1:r_k, 1:r_k}_{k+1} 
    \\
    \0^\T 
  \end{bmatrix},
  \\[2ex]
  \Mr{S}_{k+1}
  & =
    \diag(s_{1,k+1}, \dots, s_{r_k,k+1}),
  \\[2ex]
  \Mr{D}_{k+1}
  & =
    \begin{bmatrix}
      \M{D}^{1:r_k, 1:r_k}_{k+1} 
      \\
      \M{D}^{r_k+1, 1:r_k}_{k+1} 
    \end{bmatrix}.
\end{align*}
Exploiting the latter representation, one can show that if we adopt
\begin{displaymath}
  \Mt{U}_{k+1} =  \Mt{U}_k
  \quad
  \text{and}
  \quad
  \Mt{C}_{k+1} = \Mt{C}_k \M{C}^{1:r_k, 1:r_k}_{k+1}
\end{displaymath}
for an update rule for the matrices $\Mt{U}_k$ and $\Mt{C}_k$, and if we adopt 
\begin{displaymath}
  \Mt{V}_{k+1}
  =
  \begin{bmatrix}
    \Mt{V}_k
    \\
   \M{D}^{r_k+1, 1:r_k}_{k+1}  
    \left(\Mt{D}_k \M{D}^{1:r_k,1:r_k}_{k+1}\right)^+
  \end{bmatrix}
  \quad
  \text{and}
  \quad
  \Mt{D}_{k+1}
  =
    \Mt{D}_k
  \M{D}^{1:r_k,1:r_k}_{k+1}
\end{displaymath}
for an update rule for the matrices $\Mt{V}_k$ and $\Mt{D}_k$, then
the SVD of $\M{X}_{k+1}$ is given by
\begin{displaymath}
  \M{X}_{k+1} = \Mt{C}_{k+1} \Mr{S}_{k+1} \Mt{D}^\T_{k+1}.
\end{displaymath}

\subsubsection{Explanation}

\begin{align*}
  \begin{bmatrix}
    \Mt{V}_k \Mt{D}_k & \0
    \\
    \0^\T & 1
  \end{bmatrix}
            \M{D}_{k+1}
          & =
            \begin{bmatrix}
              \Mt{V}_k \Mt{D}_k & \0
              \\
              \0^\T & 1
            \end{bmatrix}
                      \begin{bmatrix}
                        \M{D}^{1:r_k, 1:r_k}_{k+1} & \M{D}^{1:r_k, r_k+1}_{k+1}
                        \\
                        \M{D}^{r_k+1, 1:r_k}_{k+1} & \M{D}^{r_k+1, r_k+1}_{k+1}
                      \end{bmatrix}
  \\
                      & =
                        \begin{bmatrix}
                          \Mt{V}_k \Mt{D}_k \M{D}^{1:r_k, 1:r_k}_{k+1}
                          &
                           \Mt{V}_k \Mt{D}_k \M{D}^{r_k+1, 1:r_k}_{k+1} 
                          \\
                          \M{D}^{r_k+1, 1:r_k}_{k+1}
                          &
                          \M{D}^{r_k+1, r_k+1}_{k+1}
                        \end{bmatrix}
\end{align*}
Deleting the last column, we obtain the matrix
\begin{displaymath}
  \begin{bmatrix}
    \Mt{V}_k \Mt{D}_k \M{D}^{1:r_k, 1:r_k}_{k+1}
    \\
    \M{D}^{r_k+1, 1:r_k}_{k+1}
  \end{bmatrix},
\end{displaymath}
which we represent as
\begin{align*}
  \begin{bmatrix}
    \Mt{V}_k \Mt{D}_k \M{D}^{1:r_k, 1:r_k}_{k+1}
    \\
    \M{D}^{r_k+1, 1:r_k}_{k+1}
  \end{bmatrix}
  & =
  \begin{bmatrix}
    \Mt{V}_k \Mt{D}_k \M{D}^{1:r_k, 1:r_k}_{k+1}
    \\
    \M{D}^{r_k+1, 1:r_k}_{k+1}
  \end{bmatrix}
  \\
  & =
  \begin{bmatrix}
    \Mt{V}_k \Mt{D}_k \M{D}^{1:r_k, 1:r_k}_{k+1}
    \\
    \M{D}^{r_k+1, 1:r_k}_{k+1}
    \left(\Mt{D}_k \M{D}^{1:r_k,1:r_k}_{k+1}\right)^+
    \Mt{D}_k \M{D}^{1:r_k, 1:r_k}_{k+1}
  \end{bmatrix}
   \\
  & =
    \underbrace{%
  \begin{bmatrix}
    \Mt{V}_k
    \\
    \M{D}^{r_k+1, 1:r_k}_{k+1}
    \left(\Mt{D}_k \M{D}^{1:r_k,1:r_k}_{k+1}\right)^+
  \end{bmatrix}
  }_{\Mt{V}_{k+1}}
  \underbrace{%
  \Mt{D}_k \M{D}^{1:r_k, 1:r_k}_{k+1}
  }_{\Mt{D}_{k+1}}.
\end{align*}

\section{Downdating}

Let $\M{X}_0$ be a $p \times q$ matrix, and let
\begin{displaymath}
  \M{X}_0 = \M{U}_0 \M{S}_0 \M{V}_0^\T
\end{displaymath}
be the SVD of $\M{X}_0$.
For $1 \leq k \leq q$, let $\M{X}_k$ be the matrix obtained from $\M{X}_0$ by deleting the last $k$ columns,
\begin{displaymath}
  \M{X}_k = \M{X}_0^{:,1:(q -k)}.
\end{displaymath}
Let 
\begin{displaymath}
  \M{X}_k = \M{U}_k \M{S}_k\M{V}_k^\T
\end{displaymath}
be the SVD of $\M{X}_k$. We shall give a recursive formula for $\M{U}_k$, $\M{S}_k$, $\M{V}_k$, and some auxiliary matrices.

Given $n \in \N$ and $i \in \{1,2, \dots, n\}$, let $\eve{i}{n} = [0, \dots, 0, 1, 0, \dots, 0]^\T$ be the length-$n$ vector with $1$ in the $i$-th position and $0$ in all others.

For each $k = 0, 1, \dots, q-1$, let $r_k$ denote the rank of $\M{X}_k$, which is the same as the rank of $\M{S}_k$.

We shall first give a formula for the SVD of $\M{X}_1$. Let
\begin{displaymath}
  \ve{n}_1 = \M{V}_0^\T \eve{q}{q}
  \quad
  \text{and}
  \quad
  \ve{q}_1 =  \frac{\eve{q}{q} - \M{V}_0 \ve{n}_1}{\sqrt{1 - \| \ve{n}_1\|^2}}.
\end{displaymath}
We consider two cases.
\begin{enumerate}[font=\upshape,label=(\roman*),wide,align=right]
\item Suppose that $\ve{q}_1 \neq \0$. Let
  \begin{displaymath}
    \M{K}_1
    =
    \begin{bmatrix}
      \M{S}_0 & \0
      \\
      \0^\T & 0
    \end{bmatrix}
    \left(
      \M{I}
      -
      \begin{bmatrix}
        \ve{n}_1
        \\
        0
      \end{bmatrix}
      \begin{bmatrix}
        \ve{n}_1
        \\
        \sqrt{1 - \| \ve{n}_1 \|^2}
      \end{bmatrix}^\T
    \right).
  \end{displaymath}
  Let
  \begin{displaymath}
    \M{K}_1 = \M{C}_1 \M{S}_1  \M{D}_1^\T
  \end{displaymath}
  be the SVD of $\M{K}_1$.  Then the matrix $\M{C}_1$ takes the form
  \begin{displaymath}
    \M{C}_1
    =
    \begin{bmatrix}
      \M{C}_1^{1:r_0, :} 
      \\
      \0^\T 
    \end{bmatrix}.
  \end{displaymath}
  Set
  \begin{displaymath}
    \Mt{C}_1 = \M{C}^{1:r_0,:}_1,
    \quad
    \Mt{V}_1 = [\M{V}_0, \ve{q}_1]^{1:(q - 1),:}
    \quad
    \text{and}
    \quad
    \Mt{D}_1 = \M{D}_1.
  \end{displaymath}
  Here $[\M{V}_0, \ve{q}_1]^{1:(q - 1),:}$ stands for the matrix $[\M{V}_0, \ve{q}_1]$ with the last row deleted. The SVD of $\M{X}_1$ is given by
  \begin{displaymath}
    \M{X}_1 = \M{U}_1 \M{S}_1\M{V}_1^\T,
  \end{displaymath}
  where
  \begin{displaymath}
    \M{U}_1 = \M{U}_0 \Mt{C}_1
    \quad
    \text{and}
    \quad
    \M{V}_1 = \Mt{V}_1 \Mt{D}_1.
  \end{displaymath}
\item Suppose that $\ve{q}_1 = \0$. Let
  \begin{displaymath}
    \M{K}_1
    =
    \M{S}_0 - \M{S}_0 \ve{n}_1 \ve{n}_1^\T
    =
    \M{S}_0 \left( \M{I} -  \ve{n}_1 \ve{n}_1^\T \right).
  \end{displaymath}
  Let
  \begin{displaymath}
    \M{K}_1 = \M{C}_1 \M{S}_1  \M{D}_1^\T
  \end{displaymath}
  be the SVD of $\M{K}_1$.  Set
  \begin{displaymath}
    \Mt{C}_1 = \M{C}_1,
    \quad
    \Mt{V}_1 = \M{V}_0^{1:(q - 1),:}
    \quad
    \text{and}
    \quad
    \Mt{D}_1 = \M{D}_1.
  \end{displaymath}
  Here $\M{V}_0^{1:(q - 1),:}$ stands for the matrix $\M{V}_0$ with the last row deleted. The SVD of $\M{X}_1$ is given by
  \begin{displaymath}
    \M{X}_1 = \M{U}_1 \M{S}_1\M{V}_1^\T,
  \end{displaymath}
  where
  \begin{displaymath}
    \M{U}_1 = \M{U}_0 \Mt{C}_1
    \quad
    \text{and}
    \quad
    \M{V}_1 = \Mt{V}_1 \Mt{D}_1.
  \end{displaymath}
\end{enumerate}

We shall now give a formula for the SVD of $\M{X}_{k+1}$. Suppose that $\Mt{C}_k$, $\Mt{V}_k$, $\Mt{D}_k$, $\M{S}_k$, and $\M{V}_k$ are known. Let
\begin{displaymath}
  \ve{n}_{k+1} = \M{V}_k^\T \eve{q-k}{q-k}
  \quad
  \text{and}
  \quad
  \ve{q}_{k+1} =  \frac{\eve{q-k}{q-k} - \M{V}_k \ve{n}_{k+1}}{\sqrt{1 - \| \ve{n}_{k+1}\|^2}}.
\end{displaymath}
We consider two cases.
\begin{enumerate}[font=\upshape,label=(\roman*),wide,align=right]
\item Suppose that $\ve{q}_{k+1} \neq \0$. Let
\begin{displaymath}
  \M{K}_{k+1}
  =
  \begin{bmatrix}
    \M{S}_k & \0
    \\
    \0^\T & 0
  \end{bmatrix}
  \left(
    \M{I}
    -
    \begin{bmatrix}
      \ve{n}_{k+1}
      \\
      0
    \end{bmatrix}
    \begin{bmatrix}
      \ve{n}_{k+1}
      \\
      \sqrt{1 - \| \ve{n}_{k+1} \|^2}
    \end{bmatrix}^\T
  \right).
\end{displaymath}
Let
\begin{displaymath}
  \M{K}_{k+1} = \M{C}_{k+1} \M{S}_{k+1}  \M{D}_{k+1}^\T
\end{displaymath}
be the SVD of $\M{K}_{k+1}$.  Then the matrix $\M{C}_{k+1}$ takes the form
\begin{displaymath}
  \M{C}_{k+1}
  =
  \begin{bmatrix}
    \M{C}_{k+1}^{1:r_k, :} 
    \\
    \0^\T 
  \end{bmatrix}.
\end{displaymath}
Set
\begin{displaymath}
  \Mt{C}_{k+1} = \Mt{C}_k \M{C}^{1:r_k, :}_{k+1},
  \quad
  \Mt{V}_{k+1}
  =
  [\Mt{V}_k,  \ve{q}_{k+1}]^{1:(q-k-1),:},
\end{displaymath}
and
\begin{displaymath}
  \Mt{D}_{k+1}
  =
  \begin{bmatrix}
    \Mt{D}_k & \0
    \\
    \0^\T & 1
  \end{bmatrix}
  \M{D}_{k+1}.
\end{displaymath}
Here $[\Mt{V}_k, \ve{q}_{k+1}]^{1:(q - k - 1),:}$ stands for the matrix $[\Mt{V}_k, \ve{q}_{k+1}]$ with the last row deleted.  The SVD of $\M{X}_{k+1}$ is given by
\begin{displaymath}
  \M{X}_{k+1} = \M{U}_{k+1} \M{S}_{k+1}\M{V}_{k+1}^\T,
\end{displaymath}
where
\begin{displaymath}
  \M{U}_{k+1} = \M{U}_0 \Mt{C}_{k+1}
  \quad
  \text{and}
  \quad
  \M{V}_{k+1} = \Mt{V}_{k+1} \Mt{D}_{k+1}.
\end{displaymath}
\item Suppose that $\ve{q}_{k+1} =\0$. Let
\begin{displaymath}
  \M{K}_{k+1}
  = \M{S}_k - \M{S}_k \ve{n}_{k+1} \ve{n}_{k+1}^\T
  = \M{S}_k \left( \M{I} - \ve{n}_{k+1} \ve{n}_{k+1}^\T \right).
\end{displaymath}
Let
\begin{displaymath}
  \M{K}_{k+1} = \M{C}_{k+1} \M{S}_{k+1}  \M{D}_{k+1}^\T
\end{displaymath}
be the SVD of $\M{K}_{k+1}$.  
Set
\begin{displaymath}
  \Mt{C}_{k+1} = \Mt{C}_k \M{C}_{k+1},
  \quad
  \Mt{V}_{k+1}
  =
  \Mt{V}_k^{1:(q-k-1),:},
  \quad
  \text{and}
  \quad
  \Mt{D}_{k+1}
  = \Mt{D}_k \M{D}_{k+1}.
\end{displaymath}
Here $\Mt{V}_k^{1:(q - k - 1),:}$ stands for the matrix $\Mt{V}_k$ with the last row deleted.  The SVD of $\M{X}_{k+1}$ is given by
\begin{displaymath}
  \M{X}_{k+1} = \M{U}_{k+1} \M{S}_{k+1}\M{V}_{k+1}^\T,
\end{displaymath}
where
\begin{displaymath}
  \M{U}_{k+1} = \M{U}_0 \Mt{C}_{k+1}
  \quad
  \text{and}
  \quad
  \M{V}_{k+1} = \Mt{V}_{k+1} \Mt{D}_{k+1}.
\end{displaymath}

\end{enumerate}

% \begin{align*}
%   \M{C}_1
%   & =
%     \begin{bmatrix}
%       \M{C}_1^{1:r_0, :} 
%       \\
%       \0^\T 
%     \end{bmatrix},
%   % =
%   % \begin{bmatrix}
%   %   \M{C}^{1:r_0, 1:r_1} 
%   %   \\
%   %   \0^\T 
%   % \end{bmatrix},
%   \\[2ex]
%   \M{S}_1
%   & =
%     \diag(s_1^{(1)}, \dots, s_{r_1}^{(1)}),
%   \\[2ex]
%   \M{D}_1
%   & =
%     \begin{bmatrix}
%       \M{D}^{1:r_0, 1:r_1}_1
%       \\
%       \M{D}^{r_0+1, 1:r_1}_1
%     \end{bmatrix}.
% \end{align*}

\begin{comment}

\begin{displaymath}
  \M{U}_1 = \M{U}_0 \Mt{C}_1
\end{displaymath}

Let
\begin{displaymath}
  \Mt{V}_1 = [\M{V}_0, \ve{q}_1]^{1:(q_0 - 1),:}
  \quad
  \text{and}
  \quad
  \Mt{D}_1 = \M{D}_1,
\end{displaymath}

\begin{displaymath}
   \M{X}_1 = \M{U}_1 \M{S}_1 \Mt{V}_1 \Mt{D}_1.
\end{displaymath}


\begin{displaymath}
  \ve{n}_k = \M{V}_{k-1}^\T \eve{q-k+1}{q},
  \quad
  \ve{q}_k
  = \frac{\eve{q-k+1}{q} - \M{V}_{k-1} \ve{n}_k}
  {\|  \eve{q-k+1}{q} - \M{V}_{k-1} \ve{n}_k\|}
  =
  \frac{\eve{q-k+1}{q} - \M{V}_{k-1} \ve{n}_k}{\sqrt{1 - \| \ve{n}_k \|^2}}
\end{displaymath}

\begin{align*}
  \|  \eve{q-k+1}{q} - \M{V}_{k-1} \ve{n}_k\|^2
  & =
  \| \eve{q-k+1}{q} \| ^2
    - 2 (\eve{q-k+1}{q})^\T \M{V}_{k-1}  \M{V}_{k-1}^\T \eve{q-k+1}{q}
  \\
  & \quad
    + \|  \M{V}_{k-1}  \M{V}_{k-1}^\T \eve{q-k+1}{q} \|^2
\end{align*}

\begin{displaymath}
  (\eve{q-k+1}{q})^\T \M{V}_{k-1}  \M{V}_{k-1}^\T \eve{q-k+1}{q}
  =  \ve{n}_k^\T  \ve{n}_k = \| \ve{n}_k \|^2
\end{displaymath}

\begin{displaymath}
  \|  \M{V}_{k-1}  \M{V}_{k-1}^\T \eve{q-k+1}{q} \|
  =
  \| \M{V}_{k-1}^\T \eve{q-k+1}{q} \| = \|  \ve{n}_k \|.
\end{displaymath}

\begin{displaymath}
  \M{K}_k
  =
  \begin{bmatrix}
    \M{S}_{k-1} & \0
    \\
    \0^\T & 0
  \end{bmatrix}
  \left(
    \M{I}
    -
    \begin{bmatrix}
      \ve{n}_k
      \\
      0
    \end{bmatrix}
    \begin{bmatrix}
      \ve{n}_k
      \\
      \sqrt{1 - \| \ve{n}_k \|^2}
    \end{bmatrix}
    \right)
\end{displaymath}

For each $1 \leq k \leq K$, the (reduced) SVD of $\M{K}_k$ takes the form
\begin{displaymath}
  \M{K}_k = \M{C}_k \M{S}_k  \M{D}_k^\T,
\end{displaymath}
where
\begin{align*}
  \M{C}_k
  & =
  \begin{bmatrix}
    \M{C}^{1:r_k, 1:r_k}_k
    \\
    \0^\T 
  \end{bmatrix},
  \\[2ex]
  \M{S}_k
  & =
    \diag(s_{1,k}, \dots, s_{r_k,k}),
  \\[2ex]
  \M{D}_k
  & =
    \begin{bmatrix}
      \M{D}^{1:r_k, 1:r_k}_k 
      \\
      \M{D}^{r_k+1, 1:r_k}_k
    \end{bmatrix}.
\end{align*}

Let
\begin{displaymath}
  \Mt{C}_1 = \M{C}^{1:r_0,:}_1
\end{displaymath}
and, for each $1 \leq k \leq K - 1$, let
\begin{displaymath}
  \Mt{C}_{k+1} = \Mt{C}_k \M{C}^{1:r_k, :}_{k+1}.
\end{displaymath}
For each $1 \leq k \leq K - 1$, let
\begin{displaymath}
  \M{U}_k = \M{U}_0 \Mt{C}_k
\end{displaymath}


Let
\begin{displaymath}
  \Mt{U}_1 = \M{U}_0
  \quad
  \text{and}
  \quad
  \Mt{C}_1 = \M{C}^{1:r_1, 1:r_1}_1
\end{displaymath}
and, for each $1 \leq k \leq K - 1$, 
\begin{displaymath}
  \Mt{U}_{k+1} =  \Mt{U}_k
  \quad
  \text{and}
  \quad
  \Mt{C}_{k+1} = \Mt{C}_k \M{C}^{1:r_{k+1}, 1:r_{k+1}}_{k+1}.
\end{displaymath}




Let
\begin{displaymath}
  \Mt{V}_1 = [\M{V}_0, \ve{q}_1]
  \quad
  \text{and}
  \quad
  \Mt{D}_1 = \M{D}_1,
\end{displaymath}
and, for each $1 \leq k \leq K - 1$, let
\begin{displaymath}
  \Mt{V}_{k+1}
  =
  [\Mt{V}_k,  \ve{q}_{k+1}]
  \quad
  \text{and}
  \quad
  \Mt{D}_{k+1}
  =
  \begin{bmatrix}
    \Mt{D}_k & \0
    \\
    \0^\T & 1
  \end{bmatrix}
  \M{D}_{k+1}.
\end{displaymath}
Then
\begin{displaymath}
  \M{X}_k = \Mt{U}_k \M{S}_k \Mt{V}_k \Mt{D}_k.
\end{displaymath}

\end{comment}

\bibliographystyle{unsrtnat}
\bibliography{references.bib}

\end{document}

%%% Local Variables:
%%% mode: latex
%%% TeX-master: t
%%% fill-column: 2000
%%% End:
